\documentclass[11pt]{article}
\oddsidemargin 0.0in
\evensidemargin 0.0in
\topmargin -0.5in
\textwidth 6.5in
\textheight 9.0in
\usepackage{amssymb,url,amsmath,graphicx, amsthm,mathrsfs,booktabs}
\usepackage{tikz}

\newtheorem{theorem}{Theorem}[section]
\newtheorem{definition}{Definition}[section]
\newtheorem{proposition}{Proposition}[section]
\newtheorem{remark}{Remark}[section]
\newtheorem{example}{Example}[section]

\newcommand{\Q}{\mathbb{Q}}
\newcommand{\R}{\mathbb{R}}
\newcommand{\N}{\mathbb{N}}
\newcommand{\Z}{\mathbb{Z}}
\newcommand{\FF}{\mathcal F}
\newcommand{\BB}{\mathcal B}
\newcommand{\EE}{\mathcal E}
\newcommand{\CC}{\mathcal C}
\newcommand{\PP}{\mathcal P}
\newcommand{\GG}{\mathcal G}
\newcommand{\divv}{\mathrm{div}}
\newcommand{\gradd}{\mathrm{grad}}


\title{Thesis Proposal Draft}

\date{}
\author{O. Tate Kernell}

\begin{document}
\maketitle

\section{Introduction}
Tide modeling is an important component in many areas of scientific research. From coastal flooding and sediment transport to ocean circulation, the accurate modeling of tides has widespread value in the scientific community. Unstructured triangular meshes appear to be useful in modeling the ocean with finite element methods. In this thesis, we will focus primarily on the linearized rotating shallow-water equations with damping, which are used for predicting global barotropic tides. Our goal is to provide a good preconditioner for the linearized rotating shallow-water equation. However, in order to do this, we will start from the basic acoustic wave equation.
\begin{equation}{\label{acouswave}}
\begin{split}
qu_t + \nabla p = 0, \\
k^{-1}p_t+ \nabla \cdot u = 0,
\end{split}
\end{equation}
on some domain $\Omega \times [0,T] \subset \R^d \times \R$ with $d = 2,3,$ with the assumption that $\Omega$ is polyhedral. The parameter $q$, the material density, is a measurable function bounded above and below by $q_*$ and $q^*$. The parameter $k$ is the bulk modulus of compressibility, which we assume is bounded by positive $k_*$ and $k^*$. For now, we will assume $q = 1 = k$. Additionally, we impose the boundary condition $u \cdot \nu = 0$ on $\partial\Omega$ where $\nu$ is the unit outward to $\Omega$. We choose initial conditions
\begin{equation}\label{initialconditions}
\begin{split}
&p(x,0) = p_0(x) \text{ and}\\
&u(x,0) = u_0(x).
\end{split}
\end{equation}
Converting this system into weak form, we get
\begin{equation}{\label{weakformacouswave}}
\begin{split}
(u_t,v) + (\nabla p,v) &= (f,v), \:\:\: v \in H_0(\divv) \\
(p_t, w) + (\nabla \cdot u,w) &= (g,w), \:\:\: w \in L_0^2
\end{split}
\end{equation}
where $u:[0,T] \rightarrow V \equiv H_0(\divv)$ and $p:[0,T] \rightarrow W \equiv L_0^2$, along with the initial conditions \eqref{initialconditions}.
When we integrate by parts \eqref{weakformacouswave} changes to 
\begin{equation}
\begin{split}
(u_t,v) - (p,\nabla \cdot v) + \underbrace{\langle p, v \cdot \nu \rangle_{\partial\Omega}}_{=0} &= (f,v) \\
(p_t, w) + (\nabla \cdot u,w) &= (g,w),
\end{split}
\end{equation}
which leads to our final form 
\begin{equation}{\label{finalformacouswave}}
\begin{split}
(u_t,v) - (p,\nabla \cdot v) &= (f,v)\\
(p_t, w) + (\nabla \cdot u,w) &= (g,w)
\end{split}
\end{equation}
\begin{remark}
	It is important to note that $H(\divv):= \{v \in L^2,\: \divv(v) \in L^2\}$ and $H_0(\divv) := \{ v\in H(\divv),\: v \cdot \nu|_{\partial \Omega} = 0 \}$ where $\nu$ is the outward normal.
\end{remark}
The semidiscrete mixed formulation of \eqref{finalformacouswave} is to find $u_h:[0,T] \rightarrow V_h$ and $p_h:[0,T] \rightarrow W_h$ such that
\begin{equation}\label{semidiscrete}
\begin{split}
(u_{h,t},v_h) - (p_h,\nabla \cdot v_h) = (f,v_h)\\
(p_{h,t}, w_h) + (\nabla \cdot u_h,w_h) = (g,w_h)
\end{split}
\end{equation}
for all $v_h \in V_h$ and $w_h \in W_h$ where $V_h \subset V$ and $W_h \subset W$.
By discretizing \eqref{semidiscrete} with Crank Nicolson and partitioning the time interval $[0,T]$ into equal timesteps $0 \equiv t_0 < t_1 < t_2 < ... < t_N$, where $t_i = i\Delta t$, we arrive at an approximation of the solution to the semidiscrete mixed formulation. We chose Crank Nicolson primarily because it is exactly energy conserving, but also it provides the benefit of being numerically stable and having improved convergence over explicit methods. Here, $u_h(t_n) \approx u^n_h \in V_h$ and $p_h(t_n) \approx p^n_h \in W_h$
\begin{equation}{\label{cnacoustic}}
\begin{split}
\left(\frac{u^{n+1}_h-u^n_h}{\Delta t},v_h\right)-\left(\frac{p^{n+1}_h+p^n_h}{2},\nabla \cdot v_h\right) = \left(f^{n+\frac{1}{2}},v_h\right) \\
\left(\frac{p^{n+1}_h-p^n_h}{\Delta t},w_h\right)+\left(\nabla \cdot \frac{u^{n+1}_h+u^n_h}{2},w_h\right) = \left(g^{n+\frac{1}{2}},w_h\right)
\end{split}
\end{equation}
where $f^{n+\frac{1}{2}} = \frac{f(t_{n+1})+f(t_n)}{2}$ and likewise for $g$. Letting $f = 0$ and $g=0$ and multiplying by $\Delta t$, we get
\begin{equation}
\begin{split}
\left(u^{n+1}_h-u^n_h,v_h\right)-\left(\frac{\Delta t}{2}\left(p^{n+1}_h+p^n_h\right),\nabla \cdot v_h\right) = 0 \\
\left(p^{n+1}_h-p^n_h,w_h\right)+\left(\frac{\Delta t}{2}\nabla \cdot\left(u^{n+1}_h+u^n_h\right),w_h\right) = 0
\end{split}
\end{equation}
Reshuffling terms leads to 
\begin{equation}\label{discretizationfinal}
\begin{split}
\left(u^{n+1}_h,v_h\right) - \frac{\Delta t}{2}\left(p^{n+1}_h, \nabla \cdot v_h\right) &= \left(u^n_h, v_h\right) +\frac{\Delta t}{2}\left(p^n_h,\nabla \cdot v_h\right) \\
\left(p^{n+1}_h,w_h\right) + \frac{\Delta t}{2}\left(\nabla \cdot u^{n+1}_h,w_h\right) &= \left(p^n_h,w_h\right) - \frac{\Delta t}{2}\left(\nabla \cdot u^n_h,w_h\right)
\end{split}
\end{equation}
 Let $\{\phi_i\}_{i=1}^{|W_h|}$ and $\{\psi_i\}_{i=1}^{|V_h|}$ be bases for $W_h$ and $V_h$ respectively. Then we can define mass matrices  %Symplectic-mixed Kirby
\begin{equation}
\begin{split}
M_{ij} &= (\phi_j,\phi_i), \\
\tilde{M}_{ij} &= (\psi_j,\psi_i).
\end{split}
\end{equation}
We can formulate \eqref{semidiscrete} as
\begin{equation}
\begin{bmatrix}
\tilde{M} & 0\\
0 & M
\end{bmatrix} \begin{bmatrix}
u_t \\
p_t
\end{bmatrix} + \begin{bmatrix}
0 & -D^T \\
D & 0
\end{bmatrix} \begin{bmatrix}
u \\
p
\end{bmatrix} = \begin{bmatrix}
F \\
G
\end{bmatrix},
\end{equation} %ode formulation
where $D$ is the differential operator $\divv$, and $F$ and $G$ are the vectors $(f,v_h)$ and $(g,w_h)$ respectively. Additionally, we can put \eqref{discretizationfinal} into matrix form, giving us
\begin{equation}\label{coeffoperator}
\begin{bmatrix}
\tilde{M} & -\frac{\Delta t}{2}D^T \\
\frac{\Delta t}{2}D & M
\end{bmatrix}
\begin{bmatrix}
u^{n+1}_h \\
p^{n+1}_h
\end{bmatrix}
=
\begin{bmatrix}
\tilde{F}\\
\tilde{G}
\end{bmatrix}
\end{equation}
where $\tilde{F} = -\left(u^n_h, v_h\right) +\frac{\Delta t}{2}\left(p^n_h,\nabla \cdot v_h\right)$ and $\tilde{G} = -\left(p^n_h,w_h\right) - \frac{\Delta t}{2}\left(\nabla \cdot u^n_h,w_h\right)$, which gives our fully discretized system.
%coercivity proof from may 2 discussion here 
\section{Preconditioning}
%Need preconditioning intro here
%here new cn operator
Now we will return to \eqref{acouswave}. If we discretize with Crank Nicolson we can express the coefficient operator as
\begin{equation}
\mathscr{A} = \begin{pmatrix}
I && \epsilon\,\gradd \\
\epsilon\,\divv && I
\end{pmatrix}
\end{equation}
where $\epsilon = \frac{\Delta t}{2}$. This is a standard preconditioning approach utilizing the Riesz map derived from the problem's spaces. 
%Need Brezzi conditions here with inf sup condition
%discretize, look at pg15 stokes problem
%Should I provide analysis on A here?
Here, we claim $\mathscr{A}$ is an isomorphism mapping  $H(\divv) \times L^2$ onto $H(\divv)^*\times L^2$, its dual space. Additionally, it is bounded on $H(\divv) \times L^2$ with bounded inverse. In the view of [] a common approach to preconditioning is to think of an equivalent operator that is easier to invert. Equivalent in this sense means that $B^{-1}A$ is a nice operator from the space into itself rather than into its dual. If it is bounded in the Hilbert space, we get mesh independent eigenvalue clustering. Our goal is to find a preconditioner $\mathscr{B}$ which maps $H(\divv)^* \times L^2$ onto $H(\divv) \times L^2$. This preconditioner will be explored below. However, we can also formulate this problem in an alternative way to be on the space $L^2 \times H^1$. This method is based on the Schur complement. %struggling here
Instead, we can implement hybridization to provide our solution. We will compare this to our explored method in a later section.
%riesz map gets rid of the mesh dependence but not the parameter dependence - kirby
%we want to adjust beta so that B remains easy to invert but the bounds on B^-1A are as independent of eps as possible. then we give an equivalent riesz map that defines a topology in which the preconditioned system has parameter independent bounds
%examples go here/citations - parameter robust preconditioners


 %two different solves: operator interpreted in 2 ways: hdiv l2 - ams (ours), l2 h1 - not actually building preconditioner - hybridize - come up with h1 discretization, sym pos def for a finite element space that lives on the edges - look at paper - equivalence between hybrid mixed and nonconforming - crouseix raviart, only continuous at edge midpoints, can plan h1 on tv (brenner and scott)

%like to get eigenvalue bounds when we go beyond this into the next?
%expand

From methods described in Mardal/Winther, we want our preconditioner to be a block diagonal operator suggested by the mapping properties of the coefficient operator of the system. The canonical preconditioner for our specific coefficient operator is derived from the spaces it maps and is as follows:
\begin{equation}
\mathscr{B}=\begin{bmatrix}
 \beta\, I -\alpha\,\gradd\,\divv& 0 \\
0 & \gamma\, I
\end{bmatrix}
\end{equation}
Here, $\mathscr{B}$ is the Riesz map, that maps the dual space back to our original space.
(Need to talk about stable finite element discretizations and stability following from inf/sup condition).
Similarly, the discrete preconditioner is of the form
\begin{equation}
\mathscr{B}_h=\begin{bmatrix}
\alpha(\nabla \cdot, \nabla \cdot) + \beta(\cdot,\cdot) & 0 \\
0 & \gamma(\cdot, \cdot)
\end{bmatrix}
\end{equation} For this preconditioner, our goal is to look at its eigenvalues as a function of $\alpha$, $\beta$, and $\gamma$. We want to choose these parameters so that our preconditioner is robust in terms of $\Delta t$. First of all, we plan on proving a theorem about the boundedness of this preconditioner. Next, we hope to show how the condition number varying with respect to $\Delta t$, $\alpha$, $\beta$, and $\gamma$. Lastly, we will prove a theorem showing that with the correct choice of $\alpha$, $\beta$, and $\gamma$, we can control the condition number so it is of order 1.
Note that our coefficient operator $\mathscr{A}$  is a bounded map with bounded inverse from $H(\divv) \times L^2$ into its dual. We can then premultiply with $\mathscr{B}$, the Riesz map, thus giving  that $\mathscr{B}^{-1}\mathscr{A}$ is a bounded operator. Additionally, this operator will be mesh independent.
%Mesh independent discretization by applying riesz map (show by kirby paper) but doesn't make it parameter robust.
%discretizing differential operator with parameters on a fixed mesh. As we change the parameters the hilbert space condition number of problem blow up, coercivity or continuity constants could get bad. Then we are farther from the riesz map.
%Building scales of spaces, defining the space to be parameter dependent, epsilon in definition of space, so a problem that may be poorly conditioned in respect to epsilon in the plain space, it may be optimally conditioned and independent of the constants in the scaled space. Inverting the riesz map in that space, we are set. Try to find the bound on the eigenvalues. Look at continuity type, infinite dimensional estimates.
 We will use this on the coefficient operator of the system \eqref{coeffoperator} and observe how well it performs. If the preconditioner is easily invertible (using AMS or others), then we have found a good preconditioner for this system.


\section{Shallow Water Equations}
 Our main goal is to create a preconditioner for the for the linearized rotating shallow-water equations with nonlinear damping. Our plan is to incrementally build up the acoustic wave equation by adding appropriate terms (damping, coriolis, etc.) and adapting the canonical preconditioner along the way. The equations are as follows
\begin{equation}
\begin{split}
u_t + \frac{f}{\epsilon}u^{\perp}+\frac{\beta}{\epsilon^2}\nabla(\eta - \eta')+g(u) &= F\\
\eta_t + \nabla \cdot (Hu) &= 0,
\end{split}
\end{equation}
where $u$ is the nondimensional two dimensional velocity field tangent to $\Omega$, $u^\perp = (-u_2,u_1)$ is the velocity rotated by $\pi/2$, $\eta$ is the nondimensional free surface elevation above the height at state of rest, $\nabla\eta'$ is the (spatially varying) tidal forcing, $\epsilon$ is the Rossby number (which is small for global tides), $f$ is the spatially-dependent non-dimensional Coriolis parameter which is equal to the sine of the latitude (or which can be approximated by a linear or constant profile for local area models), $\beta$ is the Burger number (which is also small), $H$ is the (spatially varying) nondimensional fluid depth at rest, and $\nabla$ and $\nabla \cdot$ are the intrinsic gradient and divergence operators on $\Omega$, respectively.
write out the reforming of the equations of the mixed form
and look at what the linear algebra will look like for those systems of equations and what we want to prove about them
%include energy
%Energy damping with crank nicolson
%Cfl conditions - which paper to look at?
%Riesz paper?
If the preconditioner is not easily invertible, however, we will attempt to find a more computationally suitable, spectrally equivalent preconditioner.

Of course, in order to quantify the effectiveness of our preconditioner, we plan on comparing it to other known methods. This work will be done in Firedrake. Some methods we intend to use are the Schur Complement (with Slate), AMS, and ADS.

Lastly we will ask important questions, such as:
What are the bounds on the eigenvalues of the preconditioner?
Do they behave well with respect to the parameters?
Do we have a mesh independent eigenvalue bound?
In regards to implementation, we are developing $H(\divv)$ preconditioners in Firedrake. Both our code development and numerical results will be using Firedrake and PETSc.
\section{Theorems}
\section{Preliminary Results}
Once the appropriate Riesz map preconditioner was determined for the coefficient operator in \eqref{coeffoperator}, we were able to write script in Firedrake to test some preliminary results. Using Hypre AMS, we implemented the canonical preconditioner. We compare the basic Riesz map versus hybridzation with a fixed ratio between the time step and mesh size.

\begin{table}[h!]
	\begin{center}
		\caption{Riesz Map with HypreAMS}
		\label{tab:table1}
		\begin{tabular}{l|l}
			\toprule 
			\textbf{\# Cells} & \textbf{Iterations}\\
			\midrule 
			2 & 2 \\
			4 & 6 \\
			8 &34 \\
			16 & 45 \\
			32 & 40 \\
			64 & 16 \\
			128 & 8 \\
			256 & 4 \\
			\bottomrule 
		\end{tabular}
	\end{center}
\end{table}

\begin{table}[h!]
	\begin{center}
		\caption{Preconditioning with Hybridization using SLATE}
		\label{tab:table1}
		\begin{tabular}{l|l}
			\toprule 
			\textbf{\# Cells} & \textbf{Iterations}\\
			\midrule 
			2 & 1 \\
			4 & 3 \\
			8 & 3 \\
			16 & 4 \\
			32 & 4 \\
			64 & 4 \\
			128 & 3 \\
 			256 & 3 \\
			\bottomrule
		\end{tabular}
	\end{center}
\end{table}


%preliminary results from timings compared to slate?

%\section{Future Work}
%to do future work: push through some variables
%Possible multilayer model

%mixed 
 
\end{document}
