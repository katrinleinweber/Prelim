\documentclass[11pt]{article}
\oddsidemargin 0.0in
\evensidemargin 0.0in
\topmargin -0.5in
\textwidth 6.5in
\textheight 9.0in
\usepackage{amssymb,url,amsmath,graphicx, amsthm,mathrsfs,booktabs}
\usepackage{tikz}

\newtheorem{theorem}{Theorem}[section]
\newtheorem{definition}{Definition}[section]
\newtheorem{proposition}{Proposition}[section]
\newtheorem{remark}{Remark}[section]
\newtheorem{example}{Example}[section]

\newcommand{\Q}{\mathbb{Q}}
\newcommand{\R}{\mathbb{R}}
\newcommand{\N}{\mathbb{N}}
\newcommand{\Z}{\mathbb{Z}}
\newcommand{\FF}{\mathcal F}
\newcommand{\BB}{\mathcal B}
\newcommand{\EE}{\mathcal E}
\newcommand{\CC}{\mathcal C}
\newcommand{\PP}{\mathcal P}
\newcommand{\GG}{\mathcal G}
\newcommand{\divv}{\mathrm{div}}
\newcommand{\gradd}{\mathrm{grad}}


\title{Thesis Proposal Draft}

\date{}
\author{O. Tate Kernell}

\begin{document}
\maketitle

\section{Introduction}
Tide modeling is an important component in many areas of scientific research. From coastal flooding and sediment transport to ocean circulation, the accurate modeling of tides has widespread value in the scientific community. Unstructured triangular meshes appear to be useful in modeling the ocean with finite element methods. In this thesis, we will focus primarily on the linearized rotating shallow-water equations with damping, which are used for predicting global barotropic tides. Our goal is to provide a good preconditioner for the linearized rotating shallow-water equation. We start by incrementally building up the acoustic wave equation by adding appropriate terms (damping, Coriolis, etc.) and adapting the canonical preconditioner along the way. The shallow-water equations are as follows
\begin{equation}
\begin{split}
u_t + \frac{f}{\epsilon}u^{\perp}+\frac{\beta}{\epsilon^2}\nabla(\eta - \eta')+g(u) &= F\\
\eta_t + \nabla \cdot (Hu) &= 0,
\end{split}
\end{equation}
where $u$ is the non-dimensional two dimensional velocity field tangent to $\Omega$, $u^\perp = (-u_2,u_1)$ is the velocity rotated by $\pi/2$, $\eta$ is the non-dimensional free surface elevation above the height at state of rest, $\nabla\eta'$ is the (spatially varying) tidal forcing, $\epsilon$ is the Rossby number (which is small for global tides), $f$ is the spatially-dependent non-dimensional Coriolis parameter which is equal to the sine of the latitude (or which can be approximated by a linear or constant profile for local area models), $\beta$ is the Burger number (which is also small), $H$ is the (spatially varying) nondimensional fluid depth at rest, $g(u)$ is the monotonic damping function, and $\nabla$ and $\nabla \cdot$ are the intrinsic gradient and divergence operators on $\Omega$, respectively. However, in order to do this, we will start from the basic acoustic wave equation.
\begin{equation}{\label{acouswave}}
\begin{split}
qu_t + \nabla p &= 0, \\
k^{-1}p_t+ \nabla \cdot u &= 0,
\end{split}
\end{equation}
on some domain $\Omega \times [0,T] \subset \R^d \times \R$ with $d = 2,3,$ with the assumption that $\Omega$ is polyhedral. The parameter $q$, the material density, is a measurable function bounded above and below by $q_*$ and $q^*$. The parameter $k$ is the bulk modulus of compressibility, which we assume is bounded by positive $k_*$ and $k^*$.
\begin{remark}
	It is important to note that $H(\divv):= \{v \in L^2,\: \divv(v) \in L^2\}$ and $H_0(\divv) := \{ v\in H(\divv),\: v \cdot \nu|_{\partial \Omega} = 0 \}$ where $\nu$ is the outward normal. Also note, $L^2_0$ is the space $L^2$ with zero mean.
\end{remark}
For now, we will assume $q = 1 = k$. Additionally, we impose the boundary condition $u \cdot \nu = 0$ on $\partial\Omega$ where $\nu$ is the unit outward normal to $\Omega$. We choose initial conditions
\begin{equation}\label{initialconditions}
\begin{split}
&p(x,0) = p_0(x) \text{ and}\\
&u(x,0) = u_0(x).
\end{split}
\end{equation}
Converting this system into weak form, we get
\begin{equation}{\label{weakformacouswave}}
\begin{split}
(u_t,v) + (\nabla p,v) &= (f,v), \:\:\: v \in H_0(\divv) \\
(p_t, w) + (\nabla \cdot u,w) &= (g,w), \:\:\: w \in L_0^2
\end{split}
\end{equation}
where $u:[0,T] \rightarrow V \equiv H_0(\divv)$ and $p:[0,T] \rightarrow W \equiv L_0^2$, %solved zero mean issue
 along with the initial conditions \eqref{initialconditions}.
When we integrate by parts \eqref{weakformacouswave} changes to 
\begin{equation}
\begin{split}
(u_t,v) - (p,\nabla \cdot v) + \underbrace{\langle p, v \cdot \nu \rangle_{\partial\Omega}}_{=0} &= (f,v), \\
(p_t, w) + (\nabla \cdot u,w) &= (g,w),
\end{split}
\end{equation}
which leads to our final form 
\begin{equation}{\label{finalformacouswave}}
\begin{split}
(u_t,v) - (p,\nabla \cdot v) &= (f,v),\\
(p_t, w) + (\nabla \cdot u,w) &= (g,w).
\end{split}
\end{equation}
The semidiscrete mixed formulation of \eqref{finalformacouswave} is to find $u_h:[0,T] \rightarrow V_h$ and $p_h:[0,T] \rightarrow W_h$ such that
\begin{equation}\label{semidiscrete}
\begin{split}
(u_{h,t},v_h) - (p_h,\nabla \cdot v_h) &= (f,v_h),\\
(p_{h,t}, w_h) + (\nabla \cdot u_h,w_h) &= (g,w_h),
\end{split}
\end{equation}
for all $v_h \in V_h$ and $w_h \in W_h$ where $V_h \subset V$ and $W_h \subset W$.
We can then partition the time interval $[0,T]$ into timesteps $0 \equiv t_0 < t_1 < t_2 < ... < t_N$, where $t_i = i\Delta t$. %FIXME: this technically isn't right, since the delta t's are all different sizes. Should I specify this, or is it commonplace enough that it's understood?[]
Applying Crank Nicolson then approximates the solution to the semidiscrete mixed formulation \eqref{semidiscrete}. We chose Crank Nicolson primarily because it is exactly energy conserving, but also it provides the benefit of being numerically stable. Here, $u_h(t_n) \approx u^n_h \in V_h$ and $p_h(t_n) \approx p^n_h \in W_h$
\begin{equation}{\label{cnacoustic}}
\begin{split}
\left(\frac{u^{n+1}_h-u^n_h}{\Delta t},v_h\right)-\left(\frac{p^{n+1}_h+p^n_h}{2},\nabla \cdot v_h\right) &= \left(f^{n+\frac{1}{2}},v_h\right), \\
\left(\frac{p^{n+1}_h-p^n_h}{\Delta t},w_h\right)+\left(\nabla \cdot \frac{u^{n+1}_h+u^n_h}{2},w_h\right) &= \left(g^{n+\frac{1}{2}},w_h\right),
\end{split}
\end{equation}
where $f^{n+\frac{1}{2}} = \frac{f(t_{n+1})+f(t_n)}{2}$ and likewise for $g$. Letting $f = 0$ and $g=0$ to show that energy is conserved and multiplying by $\Delta t$, we get
\begin{equation}
\begin{split}
\left(u^{n+1}_h-u^n_h,v_h\right)-\left(\frac{\Delta t}{2}\left(p^{n+1}_h+p^n_h\right),\nabla \cdot v_h\right) &= 0 \\
\left(p^{n+1}_h-p^n_h,w_h\right)+\left(\frac{\Delta t}{2}\nabla \cdot\left(u^{n+1}_h+u^n_h\right),w_h\right) &= 0,
\end{split}
\end{equation}
Reshuffling terms in \eqref{cnacoustic} leads to 
\begin{equation}\label{discretizationfinal}
\begin{split}
\left(u^{n+1}_h,v_h\right) - \frac{\Delta t}{2}\left(p^{n+1}_h, \nabla \cdot v_h\right) &= \tilde{F}, \\
\left(p^{n+1}_h,w_h\right) + \frac{\Delta t}{2}\left(\nabla \cdot u^{n+1}_h,w_h\right) &= \tilde{G},
\end{split}
\end{equation}
where
\begin{eqnarray}
\begin{split}
\tilde{F} &= \left(u^n_h, v_h\right) +\frac{\Delta t}{2}\left(p^n_h,\nabla \cdot v_h\right)+\left(f^{n+\frac{1}{2}},v_h\right), \text{ and }\\
\tilde{G} &= \left(p^n_h,w_h\right) - \frac{\Delta t}{2}\left(\nabla \cdot u^n_h,w_h\right)+\left(g^{n+\frac{1}{2}},w_h\right).
\end{split}
\end{eqnarray}
 Let $\{\phi_i\}_{i=1}^{|W_h|}$ and $\{\psi_i\}_{i=1}^{|V_h|}$ be bases for $W_h$ and $V_h$ respectively. Then we can define mass matrices  %Symplectic-mixed Kirby
\begin{equation}
\begin{split}
M_{ij} &= (\phi_j,\phi_i), \\
\tilde{M}_{ij} &= (\psi_j,\psi_i).
\end{split}
\end{equation}
We can formulate \eqref{semidiscrete} as
\begin{equation}
\begin{bmatrix}
\tilde{M} & 0\\
0 & M
\end{bmatrix} \begin{bmatrix}
u_t \\
p_t
\end{bmatrix} + \begin{bmatrix}
0 & -D^T \\
D & 0
\end{bmatrix} \begin{bmatrix}
u \\
p
\end{bmatrix} = \begin{bmatrix}
F \\
G
\end{bmatrix},
\end{equation} %ode formulation
where
\begin{equation}
D_{ij} = (\divv \psi_i, \phi_j),
\end{equation}
 is the discrete div operator and $F$ and $G$ are the vectors $(f,v_h)$ and $(g,w_h)$ respectively. Using these same matrices, we can put them into block matrix form, giving us
\begin{equation}\label{FEcoeffoperator}
\mathscr{A}_h
\begin{bmatrix}
u^{n+1}_h \\
p^{n+1}_h
\end{bmatrix}
=
\begin{bmatrix}
\tilde{F}\\
\tilde{G}
\end{bmatrix},
\end{equation}
where
\begin{equation}\label{discretcoefop}
\begin{split}
\mathscr{A}_h &= \begin{bmatrix}
\tilde{M} & -\frac{\Delta t}{2}D^T \\
\frac{\Delta t}{2}D & M
\end{bmatrix},
\end{split}
\end{equation} which gives our fully discretized system.

We want to show our matrix $\mathscr{A}_h$ is bounded and invertible. Additionally, we will investigate this question: As the mesh is refined, will the scale of inverse matrices be uniformly bounded in norm? We see that by stripping off the block diagonal we are left with a skew perturbation of a SPD matrix.
%Now we must consider \eqref{discretizationfinal}'s uniqueness and existence. For simplicity's sake, we will drop the superscripts and subscripts. %By letting $v = p$ and $w=u$
%FIXME: Not sure what happened to part of the above paragraph
Another question we must address is the coercivity of \eqref{FEcoeffoperator}. %I have this partially worked out on paper but I'd like to talk about it when we meet next
Looking at \eqref{discretizationfinal}, we can view that system of two bilinear forms as a bilinear form on the cartesian product, test and trial are pairs. 
\begin{equation}\label{bilinear}
a((u,p),(v,w)) = (u_h^{n+1},v_h) - k(p_h^{n+1}, \nabla \cdot v_h) + k(\nabla \cdot u_h^{n+1},w_h) + (p_h^{n+1},w_h) = a(U,V),
\end{equation}
where $U = (u,p)$, $V = (v,w)$, and $k = \frac{\Delta t}{2}$. Following a coercivity proof, we see
\begin{equation}
a(U,U) = \|u_h^{n+1}\|^2 + \|p_h^{n+1}\|^2,
\end{equation}
since the middle two terms cancel each other out. However, this only proves coercivity in $(L^2)^2 \times L^2$, which is not what we need.
By applying Cauchy-Schwarz [] and the Inverse Inequality [] we see
\begin{equation}
\begin{split}
a(U,V) &\leq \|u_h^{n+1}\|\|v_h\| + \frac{kC_I}{h}\|p_h^{n+1}\|\|v_h\| + \frac{kC_I}{h}\|u_h^{n+1}\|\|w_h\| + \|p_h^{n+1}\|\|w_h\| \\
&\leq \left(2 + \frac{2kC_I}{h}\right)\|U\|\|V\|.
\end{split}
\end{equation}
%works for any theta method? could add that somewhere 20180808
If we fix the time step, the condition number grows as $h \rightarrow 0$. However, if we hold $\frac{k}{h}$ fixed. $\frac{k}{h}$ is order 1 and it is continuous and coercive on $L^2$. As we take bigger time steps we should see growth in iteration count if we use the Riesz map.
%parameter dependent estimates on l2
%if we want it robust with k, k can't be tied to h, must be hdiv l2, then we want to tweak hdiv l2 parameters
%maybe need to add this to preliminary results
%continuous with constant dependent on delta t
%ADD: need to figure out Hdiv L2
%not going to find bound dependent of timestep for L2 coercivity, could just inverst block diagonal mass matrix which is the Riesz Map Preconditioner in L2
%L2 riesz not coercive on Hdiv l2, but bounded with bounded inverse
%Notes 0524 - this is for poisson - finite dimensional so uniqueness implies existence - pick v = u, w = p \implies (u,u) = 0, implies u = 0 - only solution to this system
%how to get p? if u=0, since divergence is an onto operator, then (p,div v) = 0, therefore there exists a v in Vh such that div v = p, so pick it and thus p = 0 as well. Therefore forevery Vh Wh there exists an unique solution. Now we need a uniform bound on it, how big is u, how big is p
%||u||^2 = (g,u) imples ||u|| \leq ||g||
%  part of the inverse is bounded through 
\section{Preconditioning}
%ADD: Need preconditioning intro here
Returning to the PDE \eqref{acouswave} with coefficients equal to 1, if we apply Crank Nicolson in the time derivative without discretizing in space, it becomes
\begin{equation}
\begin{split}
\frac{u^{n+1}-u^n}{\Delta t} + \nabla \left( \frac{1}{2}\left( p^{n+1} + p^n \right) \right) &= 0,\\
\frac{p^{n+1} -p^n}{\Delta t} + \nabla \cdot \left( \frac{1}{2} \left( u^{n+1}+u^n \right)\right) &=0,
\end{split}
\end{equation}
which leads to
\begin{equation}
\begin{split}
u^{n+1} + \frac{\Delta t}{2}\nabla p^{n+1} &= u^n - \frac{\Delta t}{2}\nabla p^n\\
p^{n+1} + \frac{\Delta t}{2}\nabla \cdot u^{n+1} &= p^n - \frac{\Delta t}{2}\nabla \cdot u^n.
\end{split}
\end{equation}
Therefore, at each timestep we have a discretization of the coefficient operator $\mathscr{A}$, described as
%discretize in time to pde
\begin{equation}
\mathscr{A} = \begin{pmatrix}
I && \epsilon\,\gradd \\
\epsilon\,\divv && I
\end{pmatrix}
\end{equation} %need to think about what kind of operator this is and how we can create a robust preconditioner for it
%ADD: show/cite bounded with bounded inverse
where $\epsilon = \frac{\Delta t}{2}$.  By discretizing in the finite element space $H_0(\divv) \times L_0^2$, as seen in \eqref{FEcoeffoperator}, we recover our finite dimensional coefficient operator, $\mathscr{A}_h$, defined in \eqref{discretcoefop}.
%at each timestep we need to invert a discretization of A, we need to figure out what the right differential operator to use as a preconditioner is, and discretize it
%Need Brezzi conditions here with inf sup condition possibly
%Should I provide analysis on A here?
%ADD: A is elliptic
We claim $\mathscr{A}$ is an isomorphism mapping  $H(\divv) \times L^2$ onto $H(\divv)^*\times L^2$, its dual space. In the view of [] a common approach to preconditioning is to create an equivalent operator that is easier to invert numerically. Equivalent in this sense means that $\mathscr{B}^{-1}\mathscr{A}$ is a nice operator from the initial space into itself rather than into its dual, where $\mathscr{B}$ is the preconditioner. If $\mathscr{B}^{-1}\mathscr{A}$ is bounded in the Hilbert space, we get mesh independent eigenvalue clustering []. Our goal is to find a preconditioner $\mathscr{B}$ which maps $H(\divv)^* \times L^2$ onto $H(\divv) \times L^2$. This preconditioner will be explored below. However, we can also formulate this problem in an alternative way to be on the space $L^2 \times H^1$. This method is based on the Schur complement.
%ADD: schur complement - 0517 notes
Instead, we can implement hybridization to provide our solution. We will compare this to our explored method in a later section.
%ADD: The Riesz map gets rid of mesh dependence but not parameter dependence - [kirby]. [mard/win] eps is paramter, Want to adjust B so it is still easy to invert but the bounds on B^-1A are as independent of eps as we can make them.
%we want to adjust beta so that B remains easy to invert but the bounds on B^-1A are as independent of eps as possible. then we give an equivalent riesz map that defines a topology in which the preconditioned system has parameter independent bounds
%ADD: examples go here/citations - parameter robust preconditioners
%ADD: jeonghun lee - poroelasticity - examples - 
 %two different solves: operator interpreted in 2 ways: hdiv l2 - ams (ours), l2 h1 - not actually building preconditioner - hybridize - come up with h1 discretization, sym pos def for a finite element space that lives on the edges - look at paper - equivalence between hybrid mixed and nonconforming - crouseix raviart, only continuous at edge midpoints, can plan h1 on tv (brenner and scott)

%like to get eigenvalue bounds when we go beyond this into the next?

From methods described in Mardal/Winther, we want our preconditioner to be a block diagonal operator suggested by the mapping properties of the coefficient operator of the system. The preconditioner for our specific coefficient operator $\mathscr{A}$ utilizes the Riesz map and is derived from the problem's spaces as seen below
\begin{equation}
\mathscr{B}=\begin{bmatrix}
 \beta\, I -\alpha\,\gradd\,\divv& 0 \\
0 & \gamma\, I
\end{bmatrix}.
\end{equation}
Here, if $\alpha = \beta = \gamma = 1$, $\mathscr{B}$ is the canonical Riesz map preconditioner that maps the dual space back to our original space. %bounded operator with bounded inverse on H(div) times L2
%ADD: (Need to talk about stable finite element discretizations and stability following from inf/sup condition).
Similarly, the discrete preconditioner is of the form
\begin{equation}
\mathscr{B}_h=\begin{bmatrix}
\alpha(\nabla \cdot, \nabla \cdot) + \beta(\cdot,\cdot) & 0 \\
0 & \gamma(\cdot, \cdot)
\end{bmatrix}.
\end{equation}
For this preconditioner, our goal is to look at the eigenvalues of $B_h^{-1}A_h$ as a function of $\alpha$, $\beta$, and $\gamma$. We want to choose these parameters so that our preconditioner is robust in terms of $\Delta t$ and $h$, thus eliminating the dependence on $\epsilon$. First of all, we plan on proving a theorem about the boundedness of this preconditioner and its inverse. Next, we hope to show how the condition number varies with respect to $\Delta t$, $\alpha$, $\beta$, and $\gamma$. Lastly, we will prove a theorem showing that with the correct choice of $\alpha$, $\beta$, and $\gamma$, we can control the condition number so it is of order 1.
%think about the schur compliment - wrong way - other riesz map - a bit of diffusion coeff in each off diagonal entry 
%should think about what is the hdiv - 2nd order hdiv operator it's trying to discretize - What is the schur compliment eliminating p? - probably grad of div
%doing that tells us its a 2nd order thing - gives us exactly what we need to do
Note that our coefficient operator $\mathscr{A}$  is a bounded map with bounded inverse from $H(\divv) \times L^2$ into its dual. We can then premultiply with $\mathscr{B}$, the Riesz map, thus giving  that $\mathscr{B}^{-1}\mathscr{A}$ is a bounded operator. Our goal is then to find a ball that bounds the eigenvalues of our operator regardless of the mesh refinement. Additionally, we would like to manipulate $\alpha,$ $\beta,$ and $\gamma$ so the ball is also independent of the size of the time step.
%parameters chosen correctly make preconditioner as robus to time and mesh as possible
%Mesh independent discretization by applying riesz map (show by kirby paper) but doesn't make it parameter robust.
%discretizing differential operator with parameters on a fixed mesh. As we change the parameters the hilbert space condition number of problem blow up, coercivity or continuity constants could get bad. Then we are farther from the riesz map.
%Building scales of spaces, defining the space to be parameter dependent, epsilon in definition of space, so a problem that may be poorly conditioned in respect to epsilon in the plain space, it may be optimally conditioned and independent of the constants in the scaled space. Inverting the riesz map in that space, we are set. Try to find the bound on the eigenvalues. Look at continuity type, infinite dimensional estimates.
 We will use this on the coefficient operator of the system \eqref{FEcoeffoperator} and observe how well it performs. If the preconditioner is easily invertible (using AMS or others), then we have found a good preconditioner for this system.


\section{Shallow Water Equations}
 Our main goal is to create a preconditioner for the for the linearized rotating shallow-water equations with nonlinear damping.
%write out the reforming of the equations of the mixed form
%and look at what the linear algebra will look like for those systems of equations and what we want to prove about them
%ADD: include energy
%Energy damping with crank nicolson
%Cfl conditions - which paper to look at?
%Riesz paper?
If the preconditioner is not easily invertible, however, we will attempt to find a more computationally suitable, spectrally equivalent preconditioner.

Of course, in order to quantify the effectiveness of our preconditioner, we plan on comparing it to other known methods. This work will be done in Firedrake. Some methods we intend to use are the Schur Complement (with Slate), AMS, and ADS.

Lastly we will ask important questions, such as:
What are the bounds on the eigenvalues of the preconditioner?
Do they behave well with respect to the parameters?
Do we have a mesh independent eigenvalue bound?
In regards to implementation, we are developing $H(\divv)$ preconditioners in Firedrake. Both our code development and numerical results will be using Firedrake and PETSc.
%\section{Theorems}
\section{Preliminary Results}
Once the appropriate Riesz map preconditioner was determined for the coefficient operator in \eqref{coeffoperator}, we were able to write script in Firedrake to test some preliminary results. Using Hypre AMS, we implemented the canonical preconditioner. We compare the basic Riesz map versus hybridzation with a fixed ratio between the time step and mesh size.
%What is hypreAMS?
\newpage
\begin{table}[h!]
	\begin{center}
		\caption{Riesz Map with HypreAMS}
		\label{tab:table1}
		\begin{tabular}{l|l}
			\toprule 
			\textbf{\# Cells} & \textbf{Iterations}\\
			\midrule 
			2 & 2 \\
			4 & 6 \\
			8 &34 \\
			16 & 45 \\
			32 & 40 \\
			64 & 16 \\
			128 & 8 \\
			256 & 4 \\
			\bottomrule 
		\end{tabular}
	\end{center}
\end{table}

\begin{table}[h!]
	\begin{center}
		\caption{Preconditioning with Hybridization using SLATE}
		\label{tab:table1}
		\begin{tabular}{l|l}
			\toprule 
			\textbf{\# Cells} & \textbf{Iterations}\\
			\midrule 
			2 & 1 \\
			4 & 3 \\
			8 & 3 \\
			16 & 4 \\
			32 & 4 \\
			64 & 4 \\
			128 & 3 \\
 			256 & 3 \\
			\bottomrule
		\end{tabular}
	\end{center}
\end{table}


%preliminary results from timings compared to slate?

%\section{Future Work}
%to do future work: push through some variables
%Possible multilayer model

%mixed 
 
\end{document}
